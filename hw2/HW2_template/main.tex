% Homework template for Inference and Information
% UPDATE: September 26, 2017 by Xiangxiang
\documentclass[a4paper]{article}
\usepackage{ctex}
\usepackage{amsmath, amssymb, amsthm}
\usepackage{moreenum}
\usepackage{mathtools}
\usepackage{url}
\usepackage{bm}
\usepackage{enumitem}
\usepackage{graphicx}
\usepackage{listings}
\usepackage{color}

\usepackage{fontspec}
\usepackage{xcolor}

\usepackage{float}

\definecolor{codekeyword}{RGB}{171, 0, 216}
\definecolor{codetypename}{RGB}{29, 37, 251}
\definecolor{codevariable}{RGB}{10, 23, 126}
\definecolor{codestring}{RGB}{157, 0, 25}
\definecolor{codecomment}{RGB}{31, 129, 19}
\definecolor{codebackground}{RGB}{230 235 245}

\newfontfamily\cascadia[Ligatures=ResetAll]{Cascadia Code}

\lstset{
    basicstyle          =   \small\cascadia,          % 基本代码风格
    keywordstyle        =   \bfseries,          % 关键字风格
    commentstyle        =   \rmfamily\itshape,  % 注释的风格,斜体
    stringstyle         =   \ttfamily,  % 字符串风格
    flexiblecolumns,                % 别问为什么,加上这个
    numbers             =   left,   % 行号的位置在左边
    showspaces          =   false,  % 是否显示空格,显示了有点乱,所以不现实了
    numberstyle         =   \cascadia,    % 行号的样式,小五号,tt等宽字体
    showstringspaces    =   false,
    captionpos          =   t,      % 这段代码的名字所呈现的位置,t指的是top上面
    frame               =   lrtb,   % 显示边框
}

\lstdefinestyle{Python}{
    language        =   Python, % 语言选Python
    basicstyle      =   \zihao{-5}\ttfamily,
    numberstyle     =   \zihao{-5}\ttfamily,
    keywordstyle    =   \color{blue},
    keywordstyle    =   [2] \color{teal},
    stringstyle     =   \color{magenta},
    commentstyle    =   \color{red}\ttfamily,
    breaklines      =   true,   % 自动换行,建议不要写太长的行
    columns         =   fixed,  % 如果不加这一句,字间距就不固定,很丑,必须加
    basewidth       =   0.5em,
}
\usepackage{subcaption}
\usepackage{booktabs} % toprule
\usepackage[mathcal]{eucal}
\usepackage[thehwcnt = 2]{iidef}

\setenumerate[1]{label=(\arabic{*})}
\setenumerate[2]{label=\arabic{*})}

\thecourseinstitute{清华大学电子工程系}
\thecoursename{\textbf{媒体与认知} \space 课堂2}
\theterm{2023-2024学年春季学期}
\hwname{作业}
\begin{document}
\courseheader
\name{毕嘉仪}
\vspace{3mm}
\centerline{\textbf{\Large{理论部分}}}

\section{单选题(15分)}
\subsection{\underline{C}}

\subsection{\underline{D}}

\subsection{\underline{D}}

\subsection{\underline{C}}

\subsection{\underline{B}}

\section{计算题(15 分)}
\subsection{
已知某卷积层的输入为$X$(该批量中样本数目为1,输入样本通道数为1),采用一个卷积核$W$,即卷积输出通道数为1,卷积核尺寸为$2\times 2$,卷积的步长为1,无边界延拓,偏置量为$b$:
$$X=\left[ \begin{array}{ccc}
    0.5 & -0.2 & 0.3 \\
    0.6 & 0.4 & -0.1 \\
    -0.4 & 0.5 & 0.2
\end{array}\right],
W=\left[ \begin{array}{cc}
    0.1 & -0.2  \\
    -0.3 & 0.4
\end{array}\right], b=0.04$$
}
\subsubsection{请计算卷积层的输出$Y$。}

\[y_{11} = 0.05 + 0.04 - 0.18 + 0.16 + 0.04 = 0.11\]
\[y_{12} = -0.02 - 0.02 - 0.12 - 0.04 + 0.04 = -0.20\]
\[y_{21} = 0.06 - 0.08 + 0.12 + 0.20 + 0.04 = 0.34\]
\[y_{22} = 0.04 + 0.02 - 0.15 + 0.08 + 0.04 = 0.03\]
\[\therefore Y = \begin{bmatrix}
                0.11 & -0.20 \\
                0.34 & 0.03 \\
                \end{bmatrix}\]

\subsubsection{若训练过程中的目标函数为$L$,且已知$\frac{\partial L}{\partial Y}=\left[ \begin{array}{cc}
    0.3 & 0.1 \\
    -0.4 & 0.2
\end{array} \right]$,请计算$\frac{\partial L}{\partial X}$。
}
\begin{align*}
    \frac{\partial L}{\partial X} &= \mathrm{zero\_padded}(Y) * W^{\mathrm{T}} \\
    &= \begin{bmatrix}
        0 & 0 & 0 & 0 \\
        0 & 0.11 & -0.20 & 0 \\
        0 & 0.34 & 0.03 & 0 \\
        0 & 0 & 0 & 0 \\
    \end{bmatrix} *
    \begin{bmatrix}
        0.4 & -0.3 \\
        -0.2 & 0.1 \\
    \end{bmatrix} \\
    &= \begin{bmatrix}
        0.011 & 0.104 & 0.004 \\
        0.001 & 0.039 & -0.086 \\
        -0.102 & 0.127 &0.012 \\
    \end{bmatrix}
\end{align*}

注:本题的计算方式不限,但需要提供计算过程以及各步骤的结果。
\vspace{6mm}

\centerline{\textbf{\Large{编程部分}}}
\vspace{3mm}

% 请根据是否选择自选课题的情况选择“编程作业报告”或“自选课题开题报告”中的一项完成
\section{编程作业报告}
\section{自选课题工作进度汇报}

\end{document}



%%% Local Variables:
%%% mode: late\rvx
%%% TeX-master: t
%%% End:
