% Homework template for Inference and Information
% UPDATE: September 26, 2017 by Xiangxiang
\documentclass[a4paper]{article}
\usepackage{ctex}
\usepackage{amsmath, amssymb, amsthm}
\usepackage{moreenum}
\usepackage{mathtools}
\usepackage{url}
\usepackage{bm}
\usepackage{enumitem}
\usepackage{graphicx}
\usepackage{listings}
\usepackage{color}

\lstset{
    basicstyle          =   \sffamily,          % 基本代码风格
    keywordstyle        =   \bfseries,          % 关键字风格
    commentstyle        =   \rmfamily\itshape,  % 注释的风格,斜体
    stringstyle         =   \ttfamily,  % 字符串风格
    flexiblecolumns,                % 别问为什么,加上这个
    numbers             =   left,   % 行号的位置在左边
    showspaces          =   false,  % 是否显示空格,显示了有点乱,所以不现实了
    numberstyle         =   \zihao{-5}\ttfamily,    % 行号的样式,小五号,tt等宽字体
    showstringspaces    =   false,
    captionpos          =   t,      % 这段代码的名字所呈现的位置,t指的是top上面
    frame               =   lrtb,   % 显示边框
}

\lstdefinestyle{Python}{
    language        =   Python, % 语言选Python
    basicstyle      =   \zihao{-5}\ttfamily,
    numberstyle     =   \zihao{-5}\ttfamily,
    keywordstyle    =   \color{blue},
    keywordstyle    =   [2] \color{teal},
    stringstyle     =   \color{magenta},
    commentstyle    =   \color{red}\ttfamily,
    breaklines      =   true,   % 自动换行,建议不要写太长的行
    columns         =   fixed,  % 如果不加这一句,字间距就不固定,很丑,必须加
    basewidth       =   0.5em,
}
\usepackage{subcaption}
\usepackage{booktabs} % toprule
\usepackage[mathcal]{eucal}
\usepackage[thehwcnt = 4]{iidef}

\thecourseinstitute{清华大学电子工程系}
\thecoursename{\textbf{媒体与认知}}
\theterm{2023-2024学年春季学期}
\hwname{作业}
\begin{document}
\courseheader
\name{YOUR NAME}
\vspace{3mm}
\centerline{\textbf{\Large{理论部分}}}

\section{单选题(15分)}
\subsection{\underline{?}}

\subsection{\underline{?}}

\subsection{\underline{?}}

\subsection{\underline{?}}

\subsection{\underline{?}}

\section{计算题(15 分)}
% 计算题1
\subsection{隐含马尔可夫模型}

\hspace{2em}暑假中,小E每天进行一项体育活动,包括跑步(R)、游泳(S)和打球(B),所选择的体育活动受某种潜在因素(如心情)的影响。小E每天把进行体育活动的照片发至微信朋友圈,我们可以根据观测信息推测该潜在因素的状态。

\hspace{2em}假设该潜在因素分为$S_1$和$S_2$两种状态。在$S_1$时,小E选择三种体育活动的概率分别为0.6,0.2,0.2;在$S_2$时,小E选择三种体育活动的概率分别为0.1,0.6,0.3。

\hspace{2em}该潜在因素的变化也有一定规律,若某天处于$S_1$的状态,第二天处于$S_1$和$S_2$的状态的概率分别为0.5,0.5;若某天处于$S_2$的状态,第二天处于$S_1$和$S_2$的状态的概率分别为0.6,0.4。

\hspace{2em}暑假第一天处于$S_1$和$S_2$的状态的概率均为0.5。

\vspace{3mm}
(1) 采用隐含马尔可夫模型(HMM)对小E暑假体育活动安排进行建模,{\color{blue}请写出HMM对应的参数$\lambda=\{\pi, A, B\}$}。

\vspace{3mm}
(2) 假设暑假第1、2、3天小E所进行的体育活动依次为跑步(R)、打球(B)和游泳(S),{\color{blue}请计算出现该观测序列的概率}。

\vspace{3mm}
(3) 在(2)的条件下。{\color{blue}请利用Viterbi算法推测暑假第1、2、3天最可能的隐含状态序列}。


% 请根据是否选择自选课题的情况选择“编程作业报告”或“自选课题开题报告”中的一项完成
\section{编程作业报告}
\section{自选课题工作进度汇报}

\end{document}



%%% Local Variables:
%%% mode: late\rvx
%%% TeX-master: t
%%% End:
