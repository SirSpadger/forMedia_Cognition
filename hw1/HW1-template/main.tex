% Homework Template
\documentclass[a4paper]{article}
\usepackage{ctex}
\usepackage{amsmath, amssymb, amsthm}
\usepackage{moreenum}
\usepackage{mathtools}
\usepackage{url}
\usepackage{bm}
\usepackage{enumitem}
\usepackage{graphicx}
\usepackage{subcaption}
\usepackage{booktabs} % toprule
\usepackage[mathcal]{eucal}
\usepackage[thehwcnt = 1]{iidef}

\DeclareMathOperator{\arctanh}{arctanh}

\thecourseinstitute{清华大学电子工程系}
\thecoursename{\textbf{媒体与认知} \space 课堂2}
\theterm{2023-2024学年春季学期}
\hwname{作业}
\begin{document}
\courseheader
% 请在YOUR NAME处填写自己的姓名
\name{毕嘉仪}
\vspace{3mm}
\centerline{\textbf{\Large{理论部分}}}

\section{单选题(15分)}
% 请在?处填写答案
\subsection{\underline{B}}

\subsection{\underline{A}}

\subsection{\underline{B}}

\subsection{\underline{A}}

\subsection{\underline{B}}

\section{计算题(15 分)}
\subsection{设隐含层为$\mathbf{z}=\mathbf{W}^T\mathbf{x}+\mathbf{b}$,其中$\mathbf{x}\in R^{(m \times 1)}$,$\mathbf{z}\in R^{(n\times 1)}$,$\mathbf{W}\in R^{(m\times n)}$,$\mathbf{b} \in R^{(n\times 1)}$均为已知,其激活函数如下:
$$\mathbf{y}=\delta(\mathbf{z})=tanh(\mathbf{z})$$
tanh表示双曲正切函数。若训练过程中的目标函数为L,且已知L对$\mathbf{y}$的导数 $\frac{\partial L}{\partial \mathbf{y}}=[\frac{\partial L}{\partial y_1},\frac{\partial L}{\partial y_2},...,\frac{\partial L}{\partial y_n}]^T$和$\mathbf{y}=[y_1,y_2,...,y_n]^T$的值。}
\subsubsection{请使用$\mathbf{y}$表示出$\frac{\partial \mathbf{y}^T}{\partial \mathbf{z}}$, 这里的$\mathbf{y}^T$ 为行向量。}
\begin{proof}[解]
    
    \[\frac{\partial \boldsymbol{y}^\mathrm{T}}{\partial \boldsymbol{z}}_{n \times n} = 
    \begin{bmatrix}
        \frac{\partial y_1}{\partial z_1} & \dots & \frac{\partial y_n}{\partial z_1} \\
        \vdots & \ddots & \vdots \\
        \frac{\partial y_1}{\partial z_n} & \dots & \frac{\partial y_n}{\partial z_n} \\
    \end{bmatrix}\]
    当$i \neq j$, 易知 $\frac{\partial y_i}{\partial z_i} = 0$ \\
    当$i = j$, 
    \[\tanh^\prime \boldsymbol{z} = 1 - \tanh ^2 \boldsymbol{z}\]
    \[\boldsymbol z = \arctanh \boldsymbol y\]
    \[\therefore \tanh^\prime \boldsymbol z = 1 - \]
\end{proof}

\subsubsection{请使用$\mathbf{y}$和$\frac{\partial L}{\partial \mathbf{y}}$表示$\frac{\partial L}{\partial \mathbf{x}}$,$\frac{\partial L}{\partial \mathbf{W}}$,$\frac{\partial L}{\partial \mathbf{b}}$。}
提示:$\frac{\partial L}{\partial \mathbf{x}}$,$\frac{\partial L}{\partial \mathbf{W}}$,$\frac{\partial L}{\partial \mathbf{b}}$与x,W,b具有相同维度。
\vspace{6mm}
\centerline{\textbf{\Large{编程部分}}}


\vspace{3mm}
% 请根据是否选择自选课题的情况选择“编程作业报告”或“自选课题开题报告”中的一项完成
\section{编程作业报告}
% 请在此处完成编程作业报告

\section{自选课题开题报告}
% 请在此处介绍自选课题

\end{document}



%%% Local Variables:
%%% mode: late\rvx
%%% TeX-master: t
%%% End:
